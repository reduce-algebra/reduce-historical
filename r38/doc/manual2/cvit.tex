\chapter[CVIT:Dirac gamma matrix traces]%
        {CVIT: Fast calculation of Dirac gamma matrix traces}
\label{CVIT}
\typeout{[CVIT:Dirac gamma matrix traces]}

{\footnotesize
\begin{center}
V. Ilyin, A. Kryukov, A. Rodionov and A. Taranov \\
Institute for Nuclear Physics \\
Moscow State University  \\
Moscow, 119899 Russia
\end{center}
}

\ttindex{CVIT}

The package consists of 5 sections, and provides an alternative to the
\REDUCE\ high-energy physics system.  Instead of being based on
$\Gamma$-matrices as a basis for a Clifford algebra, it is based on
treating $\Gamma$-matrices as 3-j symbols, as described by
Cvitanovic.

The functions it provides are the same as those of the standard
package.  It does have four switches which control its behaviour.

\noindent{\tt CVIT}\ttindex{CVIT}

If it is on then use Kennedy-Cvitanovic algorithm else use standard
facilities.

\noindent{\tt CVITOP}\ttindex{CVITOP}

Switches on Fierz optimisation.  Default is off;

\noindent{\tt CVITBTR}\ttindex{CVITBTR}

Switches on the bubbles and triangles factorisation.  The default is
on.


\noindent{\tt CVITRACE}\ttindex{CVITRACE}

Controls internal tracing of the CVIT package.  Default is off.

\begin{verbatim}
index j1,j2,j3,;

vecdim n$

g(l,j1,j2,j2,j1);


 2
n


g(l,j1,j2)*g(l1,j3,j1,j2,j3);


 2
n

g(l,j1,j2)*g(l1,j3,j1,j3,j2);


n*( - n + 2)
\end{verbatim}

