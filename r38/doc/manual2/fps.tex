\chapter[FPS: Formal power series]%
        {FPS: Automatic calculation of formal power series}
\label{FPS}
\typeout{[FPS: Formal power series]}

{\footnotesize
\begin{center}
Wolfram Koepf and Winfried Neun\\
Konrad--Zuse--Zentrum f\"ur Informationstechnik Berlin \\
Takustra\"se 7 \\
D--14195 Berlin--Dahlem, Germany \\[0.05in]
e--mail: Koepf@zib.de and Neun@zib.de
\end{center}
}

\ttindex{FPS}

This package can expand functions of certain type into
their corresponding Laurent-Puiseux series as a sum of terms of the form
\begin{displaymath}
\sum_{k=0}^{\infty} a_{k} (x-x_{0})^{k/n + s}
\end{displaymath}
where $s$ is the `shift number', $n$ is the `Puiseux number',
and $x_0$ is the `point of development'. The following types are
supported:
\begin{itemize}
\item
{\bf functions of `rational type'}, which are either rational or have a
rational derivative of some order;
\item
{\bf functions of `hypergeometric type'} where $a_{k+m}/a_k$ is a rational
function for some integer $m$, the `symmetry number';
\item
{\bf functions of `exp-like type'} which satisfy a linear homogeneous
differential equation with constant coefficients.
\end{itemize}

{\tt FPS(f,x,x0)}\ttindex{FPS} tries to find a formal power
series expansion for {\tt f} with respect to the variable {\tt x}
at the point of development {\tt x0}.
It also works for formal Laurent (negative exponents) and Puiseux series
(fractional exponents). If the third
argument is omitted, then {\tt x0:=0} is assumed.

Example: {\tt FPS(asin(x)\verb+^+2,x)} results in
\begin{verbatim}

         2*k  2*k             2  2
        x   *2   *factorial(k) *x
infsum(----------------------------,k,0,infinity)
        factorial(2*k + 1)*(k + 1)
\end{verbatim}
If possible, the output is given using factorials. In some cases, the
use of the Pochhammer symbol {\tt pochhammer(a,k)}$:=a(a+1)\cdots(a+k-1)$
is necessary.

{\tt SimpleDE(f,x)} tries to find a homogeneous linear differential
equation with polynomial coefficients for $f$ with respect to $x$.
Make sure that $y$ is not a used variable.
The setting {\tt factor df;} is recommended to receive a nicer output form.

Examples: {\tt SimpleDE(asin(x)\verb+^+2,x)} then results in
\begin{verbatim}
            2
df(y,x,3)*(x  - 1) + 3*df(y,x,2)*x + df(y,x)
\end{verbatim}

The depth for the search of a differential equation for {\tt f} is
controlled by the variable {\tt
fps\verb+_+search\verb+_+depth};\ttindex{fps\_search\_depth} higher
values for {\tt fps\verb+_+search\verb+_+depth} will increase the
chance to find the solution, but increases the complexity as well. The
default value for {\tt fps\verb+_+search\verb+_+depth} is 5.  For {\tt
FPS(sin(x\verb+^+(1/3)),x)}, or {\tt SimpleDE(sin(x\verb+^+(1/3)),x)}
{\em e.g.}, a setting {\tt fps\verb+_+search\verb+_+depth:=6} is necessary.

The output of the FPS package can be influenced by the\ttindex{TRACEFPS}
switch {\tt tracefps}.  Setting {\tt on tracefps} causes various
prints of intermediate results.

